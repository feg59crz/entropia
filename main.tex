\documentclass[12pt, a4paper]{article}
\usepackage[utf8]{inputenc}
\usepackage[T1]{fontenc}
\usepackage[english,brazil]{babel}
\usepackage{indentfirst}
\usepackage{graphicx}
\usepackage{amsmath}
\usepackage[left=3cm, top=3cm, right=2cm, bottom=2cm]{geometry}
\usepackage{setspace}
%links%
\usepackage{hyperref}
\setlength{\parindent}{1.25cm}

\begin{document}
% capa
\begin{titlepage}
\begin{center}
    \large
     {\bf Centro Estadual de Educação Tecnológica Paula Souza} \\
     {\bf Faculdade de Tecnologia Baixada Santista
Rubens Lara} \\ 
    {\bf Curso Superior de Tecnologia em Ciência de Dados} \\
    
    \vspace{215pt}
        %{\bf Matemática Básica} \\
    \vspace{10pt}
        {\Large \bf Análise da Precisão do ChatGPT na Resolução de Questões de Múltipla Escolha de Três Avaliações do ENEM: Um Estudo Comparativo com a Taxa de Acerto da Média Nacional}\\
        
    \vspace{100pt}

    \vfill
        {\large  \bf Autores} \\
        {\large   Fernando Gomes Cruz\\Mario Ambrósio\\Pedro}
    \vfill
        \textbf{{\large Santos}\\
        {\large 2023}}
        
\end{center}
\end{titlepage}

\tableofcontents

\newpage

\section{Introdução}
A aplicação da inteligência artificial na educação tem se mostrado uma área promissora, abrindo caminhos para a utilização de chatbots como ferramentas de suporte ao aprendizado. Nesse contexto, o ChatGPT, um modelo de linguagem baseado em Transformer, desponta como uma opção viável para auxiliar estudantes na resolução de questões de múltipla escolha. No entanto, é fundamental compreender até que ponto podemos confiar nas respostas fornecidas pelo ChatGPT e em que situações ele pode ser uma ferramenta eficaz para os estudantes.

O presente projeto de pesquisa tem como objetivo principal avaliar a precisão das respostas fornecidas pelo ChatGPT em comparação com as respostas corretas esperadas em questões de múltipla escolha do Exame Nacional do Ensino Médio (ENEM) para três anos consecutivos de avaliações. A pesquisa busca fornecer insights valiosos sobre a capacidade do ChatGPT em responder corretamente a questões de múltipla escolha e sua utilidade como uma ferramenta de suporte para estudantes que buscam auxílio na resolução dessas questões.

\section{Objetivos}
\subsection{Objetivo Principal}
Avaliar a precisão das respostas fornecidas pelo ChatGPT para questões de múltipla escolha em comparação com as respostas corretas esperadas no contexto do Exame Nacional do Ensino Médio (ENEM) em três anos consecutivos de avaliações.

\subsection{Objetivos Específicos}
\begin{itemize}
    \item Investigar a variação da capacidade do ChatGPT de responder corretamente a questões de múltipla escolha, considerando diferentes níveis de complexidade das perguntas, a fim de identificar possíveis influências da complexidade no desempenho do modelo.
    \item Avaliar a consistência do desempenho do ChatGPT em diferentes áreas de conhecimento, como Matemática, Ciências da Natureza, Ciências Humanas e Linguagens, para identificar possíveis pontos fortes ou limitações do modelo em relação a cada disciplina.
    \item Comparar a taxa de acerto do ChatGPT em questões de múltipla escolha do ENEM ao longo de três anos consecutivos de provas com a taxa de acerto média dos estudantes reais que realizaram o exame, a fim de estabelecer uma base de comparação direta e relevante para avaliar a eficácia do modelo em relação aos resultados obtidos pelos estudantes reais no ENEM.

\section{Justificativa}

A presente pesquisa tem como objetivo avaliar a precisão das respostas fornecidas pelo ChatGPT para questões de múltipla escolha em comparação com as respostas corretas esperadas no contexto do Exame Nacional do Ensino Médio (ENEM) em três provas de anos consecutivos. A investigação baseia-se na necessidade de compreender até que ponto o ChatGPT pode ser confiável como uma ferramenta de suporte para estudantes que buscam auxílio na resolução dessas questões.

A importância desse estudo reside na relevância social do tema. Com o avanço da tecnologia e o crescente uso de assistentes de inteligência artificial na educação, é fundamental verificar a eficácia e a confiabilidade dessas ferramentas para auxilo educacional, especialmente para exames importante como o ENEM, que tem grande impacto no acesso ao ensino superior.

Ao investigar a capacidade do ChatGPT em responder corretamente a questões de múltipla escolha, considerando diferentes níveis de complexidade e áreas de conhecimento, poderemos gerar insights valiosos para o seu uso em diferentes contextos educacionais. Essas informações serão relevantes tanto para os desenvolvedores de assistentes de IA quanto para os educadores que desejam utilizar essa tecnologia como recurso pedagógico.

Além disso, ao comparar a taxa de acerto do ChatGPT em questões do ENEM com a taxa de acerto média dos estudantes reais que fizeram o exame ao longo de três anos consecutivos, poderemos estabelecer uma base de comparação direta e relevante. Isso permitirá avaliar a eficácia do modelo em relação aos resultados obtidos por estudantes reais, contribuindo para uma análise mais precisa e fundamentada do desempenho do ChatGPT.

Considerando o estágio atual de desenvolvimento dos conhecimentos relacionados ao uso de assistentes de IA na educação e a relevância geral do tema, esta pesquisa tem o potencial de fornecer contribuições significativas. Os resultados obtidos poderão auxiliar no aprimoramento dessas ferramentas, no desenvolvimento de estratégias educacionais mais eficazes e na promoção de um ensino de qualidade, beneficiando tanto os estudantes quanto a área da educação como um todo.
\end{itemize}

\section{Metodologia}
Nesta seçfdescreveremos a metodologia a ser utilizada para realizar a avaliação do ENEM para o ChatGPT e comparar seu desempenho com a média nacional do país. Serão abordados os seguintes tópicos: seleção das questões, procedimentos de coleta de dados, critérios de avaliação e análise estatística.
\subsection{}
Para realizar a avaliação do ENEM para o ChatGPT, será feita a seleção de um conjunto de questões representativas das diferentes áreas de conhecimento abordadas no exame. As questões serão escolhidas com base nas provas anteriores do ENEM, buscando abranger os principais tópicos e níveis de dificuldade. Serão utilizadas as mesmas questões que compõem a prova do ENEM para o cálculo da média nacional.


\end{document}